\chapter{Security Considerations}
\label{cha:Security}

In this chapter we are going to discuss the cryptographic principles that ensure security. A detailed description of both algorithms can be found in Singnal's protocol \cite{x3dh} \cite{DoubleRachet}. Since our starting point was Signal and the fact that we implemented all the necessary components, we can say that our implementation is secure so long as all the parts of the X3DH are not compromised as proved by Ferran van der Have and B.J.M. Mennink \cite{X3DHproof}. In fact as shown the loss of just one of the components is not enough to compromise the communication.

\section{Integration between X3DH and Double Ratchet}
\label{sec:IntegrationWithX3DH}

Both algorithms work very well together: the outputs of the X3DH are used by the Double Ratchet:
\begin{itemize}
  \item The $SK$ output from X3DH becomes the $SK$ input to Double ratchet initialization.
  \item The $AD$ output from X3DH becomes the $AD$ input to Double Ratchet encryption and decryption.
  \item Bob's signed prekey form X3DH ($SPK_B$) becomes Bob's initial ratchet public key for double ratchet initialization.
\end{itemize}

This ensures the fact that Alice is the one always sending the first message.

\section{Security Considerations}
\label{sec:SecurityConsideration}

\subsection{Authentication}
\label{subsec:Authentication}

Before or after an X3DH key agreement, the parties may compare their identity public keys $IK_A$ and $IK_B$ through some authenticated channel. For example, they may compare public key fingerprints manually, or by scanning a QR code.

During the designing phase of our project we assumed that the server could not be impersonated by any malicious attacker and can not be tampered with thus making sure that what Alice or Bob get from the server are the correct keys. Thus, the only possible attack is an identity miss-binding; this can be achieved if Alice asks Eve for his key and Eve instead of giving Alice her key she gives Bobs thus making Alice link Bob's keys to the identity of Eve. But this goes beyond not only the projects capabilities but Singnal's protocol.

\subsection{Protocol Replay}
\label{subsec:ProtocolReplay}

One of the potential issues of X3DH is the possibility of a replay attack, but this is not possible thanks to the implementation of the one-time prekys. In Signal's documentation other proposed mitigations are: use a ratchet mechanism, keep a blacklist of observed messages, or replace old signed prekeys more rapidly.

\subsection{Deniability}
\label{subsec:Deniability}

X3DH doesn’t give either Alice or Bob a publishable cryptographic proof of the contents of their communication or the fact that they communicated.

Like in the OTR protocol \cite{OTR}, in some cases a third party that has compromised legitimate private keys from Alice or Bob could be provided a communication transcript that appears to be between Alice and Bob and that can only have been created by some other party that also has access to legitimate private keys from Alice or Bob.

\subsection{Key Compromise}
\label{subsec:KeyCompromise}

Compromise of a party's private keys has a huge impact on security, though the use of ephemeral keys and prekeys provides some mitigation.

Compromise of a party's identity private key allows impersonation of that party to others. Knowledge of a party's preky private keys may affect the security of older or newer $SK$ values, depending on many factors.

Since we implemented one-time prekeys then the compromise of identity keys and preky private keys at some future time will not compromise older $SK$ since the $OPK_B$ was deleted. Moreover, we also implemented double ratchet which replaces $SK$ with new keys to provide fresh forward secrecy.

\subsection{Server Trust}
\label{subsec:ServerTrust}

A malicious server could cause communication between Alice and Bob to fail for example by refusing to deliver messages.

The other possible issue is the inability to check in person keys, but a system similar to the one implemented by many Linux distros should suffice as integrity check for the key of the server. Once we have established a root of trust we can rely on the checks implemented by the protocol with the exception mentioned in \ref{subsec:Authentication}.

Another final possible attack is due to the limited number of $OTK$ if an attacker drains another party's one-time prekys, making them unable to create new connections. A system to replenish one-time prekys should be implemented, and the server should limit the rate keys are requested.

\subsection{Secure Deletion}
\label{subsec:SecureDeletion}

The double ratchet algorithm is designed to provide security against and attacker who records encrypted messages and then compromises the sender or receiver at a later time. This security could be deleted if deleted plaintext or keys could be recovered by an attacker with low-level access to the compromised device. But the main focus of this project is security of data in transit.

\subsection{Recovery from Compromise}
\label{subsec:RecoveryFromCompromise}

The DH ratchet is designed to recover security against a passive eavesdropper who observes encrypted messages after compromising the parties to a session. Compromise of secrete keys will:

\begin{itemize}
  \item The attacker could use the compromised keys to impersonate the compromised party.
  \item The attacker could substitute her own ratchet keys via continuous active MitM attack, to maintain eavesdropper on the compromised session.
  \item The attacker could modify a compromised party's RNG so that future ratchet private keys are predictable.
\end{itemize}

If a party suspects its keys or devices have been compromised, it must replace them immediately.

\subsection{Cryptanalysis and Ratchet Public Keys}
\label{subsec:CryptanalysisandRatchetPublicKeys}

Because all DH ratchet computations are mixed into the root key, an attacker who can decrypt a session with passive cryptanalysis, an attacker who can decrypt a session with passive cryptanalysis might lose this ability if she fails to observe some ratchet public keys.

But in our implementation cryptanalysis should not be an issue due to the use of strong encryption function like AES-GCM.

\subsection{Deletion of Skipped Message Keys}
\label{subsec:DeletionofSkippedMessageKeys}

Storing skipped message keys introduces some risks:

\begin{itemize}
  \item A malicious sender could induce recipients to store large numbers of skipped message keys, possibly causing DoS due to consuming storage space.
  \item The lost messages may have been seen by an attacker, even though they did not reach the recipient. The attacker can compromise the intended recipient at a later time to retrieve the skipped message keys.
\end{itemize}

The implemented mitigation is limiting the number of skipped messages to 1000. For the second one we created a set timer after which keys are deleted as proposed by the Signal documentation.
