\chapter{Installation and Configuration}
\label{cha:Installation}

\section{Getting Started}

\begin{enumerate}
    \item Clone the repository:

        \begin{lstlisting}
git clone https://github.com/christiansassi/advanced-programming-of-cryptographic-methods-project
        \end{lstlisting}

    \item Navigate to the project directory:
    
        \begin{lstlisting}
cd advanced-programming-of-cryptographic-methods-project
        \end{lstlisting}
\end{enumerate}

\section{Configuration}

Modify the \texttt{config.toml} file located in the \texttt{config} directory to adjust the application settings:

\begin{itemize}
    \item \texttt{server\_ip}: The IP address of the server (default: \texttt{127.0.0.1}). If you plan to use Docker (see \ref{sec:docker}), set this to \texttt{"server"}.
    \item \texttt{server\_port}: The port on which the server listens (default: \texttt{3333}).
    \item \texttt{log\_level}: The logging level (default: \texttt{info}).
\end{itemize}

\noindent
\begin{myWarning}
\textbf{Warning:} Do not modify \texttt{private\_key\_server} and \texttt{public\_key\_server}. These values are automatically generated when the server starts.
\end{myWarning}

\section{Installation with Docker (Recommended)}
\label{sec:docker}

\begin{enumerate}
    \item Verify that Docker is installed by running:

        \begin{lstlisting}
docker --version
        \end{lstlisting}

    \item Ensure the following line is present in \texttt{config.toml}:

        \begin{lstlisting}
server_ip = "server"
        \end{lstlisting}

    \item Build the Docker image:

        \begin{lstlisting}
docker compose build --no-cache
        \end{lstlisting}

    \item Run the updater to generate server keys:

        \begin{lstlisting}
docker compose run --rm updater
        \end{lstlisting}

    \item Start the server and clients:

        \begin{lstlisting}
docker compose up server client1 client2
        \end{lstlisting}

    \item Open a new terminal and attach to the TUI of \texttt{client1}:

        \begin{lstlisting}
docker container attach client1
        \end{lstlisting}

    \item Repeat the previous step for \texttt{client2}, if desired.

    \item To stop the containers, press \texttt{CTRL+C} in the terminal where you ran \texttt{docker compose up}, or run:

        \begin{lstlisting}
docker compose down
        \end{lstlisting}
\end{enumerate}

\section{Local Installation with Python}

\noindent
\begin{myWarning}
\textbf{Warning:} Building releases on macOS may encounter issues. If proceeding with this method, consider running \texttt{cargo build} manually in the \texttt{server} and \texttt{tui} directories.
\end{myWarning}

\begin{enumerate}
    \item Ensure Python is installed by running:

        \begin{lstlisting}
python --version
        \end{lstlisting}

    There is no specific Python version or dependency requirement, as Python is only used to simplify the launching process.

    \item Run the server:

        \begin{lstlisting}
python server.py
        \end{lstlisting}

    \item Run one or more clients:

        \begin{lstlisting}
python tui.py
        \end{lstlisting}
\end{enumerate}
