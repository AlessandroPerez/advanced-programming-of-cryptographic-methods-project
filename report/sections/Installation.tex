\chapter{Installation and Configuration}
\label{cha:Installation}

\section{Getting Started}

\begin{enumerate}
    \item Clone the repository:

        \begin{lstlisting}
git clone https://github.com/christiansassi/advanced-programming-of-cryptographic-methods-project
        \end{lstlisting}

    \item Navigate to the project directory:
    
        \begin{lstlisting}
cd advanced-programming-of-cryptographic-methods-project
        \end{lstlisting}
\end{enumerate}

\section{Configuration}

Modify the \href{https://github.com/christiansassi/advanced-programming-of-cryptographic-methods-project/blob/main/config/config.toml}{\texttt{config.toml}} file located in the \href{https://github.com/christiansassi/advanced-programming-of-cryptographic-methods-project/blob/main/config}{\texttt{config}} directory to adjust the application settings:

\begin{itemize}
    \item \texttt{server\_ip}: The IP address of the server (default: \texttt{server}). If you do not plan to use Docker (see Section~\ref{sec:source}), set this to \texttt{"127.0.0.1"} or another valid IP address.
    \item \texttt{server\_port}: The port on which the server listens (default: \texttt{3333}).
    \item \texttt{log\_level}: The logging level (default: \texttt{info}).
\end{itemize}

\noindent
\begin{myWarning}
\textbf{Warning:} Do not modify \texttt{private\_key\_server} and \texttt{public\_key\_server}. These values are automatically generated when the server starts.
\end{myWarning}

\section{Installation}

You can choose between installing the project using Docker (the default and recommended method) or building the executables manually.  
The Docker-based approach is recommended because it minimizes compatibility issues by encapsulating all dependencies within containers.  
Local installation is less user-friendly but still fully functional.  
All installation methods offer the same features. Choose the one that best fits your environment and preferences.

\vspace{10pt}
\noindent
\begin{myNote}
\textbf{Note:} If you prefer not to use Docker and want to avoid building everything from source, pre-built binaries are available in the \href{https://github.com/christiansassi/advanced-programming-of-cryptographic-methods-project/tree/main/target/release}{\texttt{target/release}} folder, except for macOS. If you are using macOS, you will need to build the source files manually. See Section~\ref{sec:source}.
\end{myNote}

\subsection{Installation with Docker (Recommended)}
\label{sec:docker}

\begin{enumerate}
    \item Verify that Docker is installed by running:

        \begin{lstlisting}
docker --version
        \end{lstlisting}

    \item Build the Docker image:

        \begin{lstlisting}
docker compose build --no-cache
        \end{lstlisting}

    \item Run the updater to generate server keys:

        \begin{lstlisting}
docker compose run --rm updater
        \end{lstlisting}

    \item Start the server and clients:

        \begin{lstlisting}
docker compose up server client1 client2
        \end{lstlisting}

    \item Open a new terminal and attach to the TUI of \texttt{client1}:

        \begin{lstlisting}
docker container attach client1
        \end{lstlisting}

    \item Repeat the previous step for \texttt{client2}, if desired.

    \item To stop the containers, press \texttt{CTRL+C} in the terminal where you ran \texttt{docker compose up}, or run \texttt{docker compose down}.

\end{enumerate}

\subsection{Installation from Source}
\label{sec:source}

\begin{enumerate}
    \item Ensure that Rust and Cargo are installed by running:
    
        \begin{lstlisting}
rustc --version
cargo --version
        \end{lstlisting}
    
    \item Build all components (\href{https://github.com/christiansassi/advanced-programming-of-cryptographic-methods-project/tree/main/server}{\texttt{server}}, \href{https://github.com/christiansassi/advanced-programming-of-cryptographic-methods-project/tree/main/tui}{\texttt{tui}}, \href{https://github.com/christiansassi/advanced-programming-of-cryptographic-methods-project/tree/main/config}{\texttt{updater}}) by running the following command from the root of the repository:

        \begin{lstlisting}
carbo build --release
        \end{lstlisting}

    \item The built executables will be located in the \href{https://github.com/christiansassi/advanced-programming-of-cryptographic-methods-project/tree/main/target/release}{\texttt{target/release}} folder.

        \begin{enumerate}
            \item First, update the server keys by running the \texttt{updater} executable.
            \item Then, start the server by running the \texttt{server} executable.
            \item Finally, run any number of clients (as needed) using the \texttt{tui} executable.
        \end{enumerate}

\end{enumerate}
